\documentclass[11pt,largemargins, letter]{homework}

\newcommand{\hwname}{Tobias Jakobi}
\newcommand{\hwemail}{tjakobi@arizona.edu}
\newcommand{\hwtype}{Project}
\newcommand{\hwnum}{1}
\newcommand{\hwclass}{CTS505}
\newcommand{\hwlecture}{2023}
\newcommand{\hwsection}{-10}

\usepackage[margin=1in]{geometry}


\renewcommand{\questiontype}{Task}

\begin{document}
\maketitle

The solutions for the tasks listed below should be saved in separate \texttt{.R} files, e.g. \texttt{task1.R}, \texttt{task2.R}.

Please include loaded libraries, loading of files, and all other commands in the files.

\question

  \begin{alphaparts}
    \questionpart Download a list of gene IDs with expression values and false discovery rate (FDR) columns from \texttt{https://links.jakobilab.org/cts505\_genes}.
    \questionpart What gene IDs have been provided?
    Identify the correct species based on the ID format.
  \end{alphaparts}


\question

  \begin{alphaparts}
    \questionpart Use the \texttt{biomaRt} package to retrieve gene names and Uniprot IDs for the provided gene IDs.
    \questionpart Print the top 10 highest expressed genes based on the logFC value into a new CSV file, \texttt{top10.csv}.
    Include the gene names and Uniprot IDs.
  \end{alphaparts}

\question

  \begin{alphaparts}
    \questionpart Use the \texttt{ggplot2} package to create a X-Y graph plotting \texttt{log10(FDR value)} on the Y axis and the logFC value on the X axis.
    \questionpart Set a title and descriptive X and Y axis labels and save the plot a PDF file \texttt{xy\_plot.pdf}.
  \end{alphaparts}

\end{document}